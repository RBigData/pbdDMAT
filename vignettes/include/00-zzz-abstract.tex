\begin{abstract}

With the size of data ever growing, the use of multiple processors in a single analysis becomes more and more a necessity.  
The Programming Big Data in R (pbdR) project attempts to address the \proglang{R} language's current shortcomings in parallel distributed computations.  The \pkg{pbdDMAT} package for \proglang{R} provides high level S4  classes and methods for block-cyclic distributed matrices.  These methods focus on computationally burdensome problems with block-cyclically distributed data, such as linear algebra and statistics, in such a way that the syntax for operating on new data types mimics \proglang{R}'s syntax for ordinary matrices as closely as possible.  
This allows someone already familiar with \proglang{R} syntax to achieve vast speed improvements via parallelism with a mostly already familiar syntax, and without the need to learn complicated parallel programming techniques.  Much of the heavy lifting is performed by the \pkg{pbdBASE} package, which utilizes the PBLAS and ScaLAPACK libraries.  In addition to performance improvements through parallelism, use of this system with more than one processor allows the user to break \proglang{R}'s local memory barrier, namely the requirement that a vector be indexed by a 32-bit integer, by only storing subsets of the vector on each processor.
\end{abstract}
