\section[]{Classes and Methods}
% \label{sec:introduction}
\addcontentsline{toc}{section}{\thesection. Classes and Methods}

Presently, package \pkg{pbdBASE} contains one new class, namely the class \code{ddmatrix} which stands for \textbf{d}istributed \textbf{d}ense \textbf{matrix}.  This S4 class serves as a container for a distributed matrix type, consisting of the members:

\begin{align*}
\text{\code{ddmatrix}}&=\begin{cases}
 $\textbf{Data}$ & $S4 slot containing the object's submatrix, an R matrix$\\
 $\textbf{dim}$ & $S4 slot containing the dimension of the global matrix, a numeric pair$\\
 $\textbf{ldim}$ & $S4 slot containing the dimension of the local submatrix, a numeric pair$\\
 $\textbf{bldim}$ & $S4 slot containing the ScaLAPACK blocking factor, a numeric pair$\\
 $\textbf{CTXT}$ & $S4 slot containing the BLACS context, an numeric singleton$
 \end{cases}
\end{align*}
with prototype
\begin{align*}
\text{\code{new("ddmatrix")}}&=\begin{cases}
 $\textbf{Data}$ & =\text{\code{matrix(0)}}\\
 $\textbf{dim}$ & =\text{\code{c(1,1)}}\\
 $\textbf{ldim}$ & =\text{\code{c(1,1)}}\\
 $\textbf{bldim}$ & =\text{\code{c(1,1)}}\\
 $\textbf{CTXT}$ & =0
 \end{cases}
\end{align*}
We will discuss the last two items in more detail in the later sections, particularly Section~\ref{sec:advanced} and Section~\ref{sec:devs}.
\np
In addition, the \pkg{pbdBASE} package contains new S4 methods for class \code{ddmatrix}, 
\begin{table}[h]
 \centering
\begin{tabular}{l|l}\hline\hline
\textbf{\proglang{R} S4 Method Overloading} & \textbf{New S4 Methods}\\\hline
 \code{[}, \code{[<-} & \code{submatrix()}, \code{submatrix<-}\\
 \code{length()}, \code{dim()}, \code{nrow()}, \code{ncol()} & \code{ldim()}, \code{bldim()}, \code{ctxt()} \\
 \code{as.matrix()}, \code{as.vector()} & \code{as.ddmatrix()}\\
 \code{na.exclude()} & \\
 \code{print()} & \\\hline\hline
\end{tabular}
\caption{Class \code{ddmatrix} Methods in Package \code{pbdBASE}}
\label{tab:methods}
\end{table}
many of which are listed in Table~\ref{tab:methods}.  Full information on these methods, as well as some new S3 methods and important non-method functions, is provided in the \pkg{pbdBASE} package documentation.